\documentclass[aspectratio=169]{beamer}

\mode<presentation>
{
  \usetheme{default}
  \usecolortheme{default}
  \usefonttheme{default}
  \setbeamertemplate{navigation symbols}{}
  \setbeamertemplate{caption}[numbered]
  \setbeamertemplate{footline}[frame number]  % or "page number"
  \setbeamercolor{frametitle}{fg=white}
  \setbeamercolor{footline}{fg=black}
} 

\usepackage[english]{babel}
\usepackage[utf8x]{inputenc}
\usepackage{tikz}
\usepackage{courier}
\usepackage{array}
\usepackage{bold-extra}
\usepackage{minted}
\usepackage[thicklines]{cancel}

\xdefinecolor{dianablue}{rgb}{0.18,0.24,0.31}
\xdefinecolor{darkblue}{rgb}{0.1,0.1,0.7}
\xdefinecolor{darkgreen}{rgb}{0,0.5,0}
\xdefinecolor{darkgrey}{rgb}{0.35,0.35,0.35}
\xdefinecolor{darkorange}{rgb}{0.8,0.5,0}
\xdefinecolor{darkred}{rgb}{0.7,0,0}
\definecolor{darkgreen}{rgb}{0,0.6,0}
\definecolor{mauve}{rgb}{0.58,0,0.82}

\title[2018-04-23-dianahep-numba-oamap]{Extending Numba for HEP data types}
\author{Jim Pivarski}
\institute{Princeton University -- DIANA-HEP}
\date{April 23, 2018}

\begin{document}

\logo{\pgfputat{\pgfxy(0.11, 7.4)}{\pgfbox[right,base]{\tikz{\filldraw[fill=dianablue, draw=none] (0 cm, 0 cm) rectangle (50 cm, 1 cm);}\mbox{\hspace{-8 cm}\includegraphics[height=1 cm]{princeton-logo-long.png}\includegraphics[height=1 cm]{diana-hep-logo-long.png}}}}}

\begin{frame}
  \titlepage
\end{frame}

\logo{\pgfputat{\pgfxy(0.11, 7.4)}{\pgfbox[right,base]{\tikz{\filldraw[fill=dianablue, draw=none] (0 cm, 0 cm) rectangle (50 cm, 1 cm);}\mbox{\hspace{-8 cm}\includegraphics[height=1 cm]{princeton-logo.png}\includegraphics[height=1 cm]{diana-hep-logo.png}}}}}

% Uncomment these lines for an automatically generated outline.
%\begin{frame}{Outline}
%  \tableofcontents
%\end{frame}

% START START START START START START START START START START START START START

\begin{frame}{Can we use Numba for LHC data?}
\vspace{0.5 cm}
Numba provides a smooth transition between high-level tinkering in Python and high-throughput processing, and we've seen a real-world demonstration in XENONnT.

\vspace{0.8 cm}
\textcolor{darkblue}{\Large Can we use it for LHC data?}

\vspace{0.2 cm}
\begin{itemize}
\item Must access ROOT data. \uncover<2->{\textcolor{darkorange}{Solution: uproot, as well as ROOT PR\#943, PR\#1872.}}
\item We need arbitrary length lists of particles per event, not rectangular arrays.
\item Converting all of our events to lists of namedtuples to feed to Numba would be a serious performance hit (memory and throughput).

\vspace{0.2 cm}
\uncover<3->{\textcolor{darkorange}{Solution: object-array mapping (OAMap).}}
\end{itemize}
\end{frame}

\begin{frame}{Object-array mapping (OAMap)}
\vspace{0.5 cm}
By analogy with object-relational mapping (ORM), which translates between classes in object-oriented programming (OOP) and relational tables in databases.

\vspace{0.5 cm}
OAMap translates between OOP objects and low-level, read-only arrays:

\vspace{0.2 cm}
\begin{itemize}
\item[$\rightarrow$] Physicists write arbitrary code on OOP-style objects.
\item[$\rightarrow$] Actual data are stored in arrays (e.g.~TBaskets), never deserialized into objects.
\item[$\rightarrow$] Physicist's function is translated into operations on arrays, not any intermediaries.
\end{itemize}

\vspace{0.2 cm}
\uncover<2->{\textcolor{darkblue}{This is a kind of compilation.} Last year's talks about ``Femtocode'' are this idea in a functional language, but the data representation can be handled on its own.}

\vspace{0.2 cm}
\uncover<3->{In particular, Python is a high-level language that physicists already use, and Numba provides optimized compilation through LLVM.}
\end{frame}

\begin{frame}{Performance tests {\it preceded} design}
\vspace{0.3 cm}
I wanted to make sure that committing to a library (Numba) did not give away performance from the start. I observed no loss in throughput compared to pure C.

\vspace{0.2 cm}
\begin{columns}[t]
\column{0.5\linewidth}
Single-threaded rates of millions of events per second, limited only by trig functions.

\begin{center}
\includegraphics[height=5 cm]{physical-media.pdf}
\end{center}
\column{0.43\linewidth}
Multi-threaded scaling limited only by physical memory bus bandwidth.

\vspace{-0.8\baselineskip}
\begin{center}
\includegraphics[height=5 cm]{knl-scaling.pdf}
\end{center}
\end{columns}
\end{frame}

\begin{frame}{}
\vspace{0.75 cm}\large

\begin{block}{\LARGE Moral:}
\vspace{0.75 cm}
If we can deliver arrays to main memory and the CPU cache quickly enough,

\vspace{0.35 cm}
and physicists' analyses can be expressed as array operations on sequential data,

\vspace{0.35 cm}
then it can be compiled to run at this scale.
\end{block}

\vspace{0.5 cm}
\begin{center}
\uncover<2->{\textcolor{darkorange}{\Large It's a problem of expressing OOP concepts in arrays.}}
\end{center}
\end{frame}

\begin{frame}{``General enough'' data types}
\vspace{0.35 cm}
\begin{enumerate}
\item \textcolor{darkblue}{\large Primitives:} \uncover<2->{any fixed-width data, such as numbers.}

\item \textcolor{darkblue}{\large Lists:} \uncover<3->{arbitrary-length collections of data with a given type (homogeneous).}

\item \textcolor{darkblue}{\large Unions:} \uncover<4->{set of possible types for heterogeneity (e.g.\ list of ``electrons {\it or} muons'').}

\item \textcolor{darkblue}{\large Records:} \uncover<5->{objects containing a set of typed fields (a.k.a.\ classes or structs).}

\item \textcolor{darkblue}{\large Tuples:} \uncover<6->{fixed-length collections of arbitrary-typed fields (like records with index positions instead of field names).}

\item \textcolor{darkblue}{\large Pointers:} \uncover<7->{objects identified by position in another field. Intended for linking relationships among particles, but also useful for expressing skimmed data as event lists, making non-tree data structures, emulating Arrow/Parquet's ``dictionary encoding'' of categorical strings\ldots}
\end{enumerate}

\vspace{0.2 cm}
The data types and their representations have \textcolor{darkblue}{compositional symmetry:} any type can be plugged into any other type and the array representations follow suit.
\end{frame}


\end{document}
